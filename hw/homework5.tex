% arara: xelatex
%!TEX TS-program = xelatex

\documentclass[12pt]{article}

% \usepackage[english, russian]{babel} % выбор языка для документа
% \usepackage[utf8]{inputenc} % задание utf8 кодировки исходного tex файла
% \usepackage[X2,T2A]{fontenc}        % кодировка


\usepackage{polyglossia}
\usepackage{fontspec}         % пакет для подгрузки шрифтов


\setmainlanguage{english}
\setotherlanguages{english}

% \setmainfont{Times New Roman}       % задаёт основной шрифт документа
\setmainfont{Linux Libertine O} % or Helvetica, Arial, Cambria
% why do we need \newfontfamily:
% http://tex.stackexchange.com/questions/91507/
\newfontfamily{\cyrillicfonttt}{Linux Libertine O}


\usepackage{physics}

\usepackage{verbatim}
\RequirePackage{lastpage}
\RequirePackage{epsfig}
\RequirePackage{graphicx}
\RequirePackage{float}
\RequirePackage{subfigure}
%\RequirePackage{overcite}
\RequirePackage{psfrag}
\RequirePackage{ifthen}
\RequirePackage{amsmath}
\RequirePackage{xcolor}
\RequirePackage{hyperref}
%\RequirePackage{cleveref}

\usepackage{indentfirst}


\usepackage[top=2cm, left=1.2cm, right=1.2cm, bottom=2cm]{geometry} % размер текста на странице

%\textwidth=6.5in
%\textheight=9in
%\topmargin=-0.25in
%\headheight=0in
%\headsep=0in
%\topskip=0in
%\oddsidemargin=0in
%\evensidemargin=0in

\newcommand{\PHNote}[1]{\ColorNote{red}{PH}{#1}}

%\renewcommand{\maketitle}{%
%	\begin{center}
%		\noindent{\bf  HSE and University of London Double Degree Programme in Data Science and Business Analytics}\\\vspace{10pt} 
%		\noindent{\bf \large  \textit{Statistical Methods for Market Research}, 2021-2022 }\\\vspace{1pt}
%	\end{center}
%	\textbf{\textit{Lecturer:}} Boris Demeshev.
%\textbf{\textit{Classteacher:}} Ksenia Kasianova.
%	\begin{center}
%		\noindent{\bf  \textit{Home assignment 1.} }\\
%	\end{center}
%	\vspace{0.2in}
%}

\usepackage{multirow}

\usepackage[backend=biber, style=numeric-comp,
	sorting=ynt,  
	defernumbers=true,
	 babel=other]{biblatex}

\addbibresource{references.bib}

\newcommand{\R}{\mathbb{R}}
\newcommand{\E}{\mathbb{E}}
\renewcommand{\var}{\mathbb{V}}
\newcommand{\1}{\mathbbm{I}}
\newcommand{\heta}{\hat{\eta}}
\newcommand{\hgamma}{\hat{\gamma}}
\newcommand{\Bor}{\mathcal{B}}
\newcommand{\N}{\mathbb{N}}
\newcommand{\Xtn}{\tilde{X}^n}
\newcommand{\clawn}{\underset{n\rightarrow
		+\infty}{\Longrightarrow}} 
\newcommand{\convn}{\underset{n\rightarrow
		+\infty}{\longrightarrow}} 
\newcommand{\p}{\mathbb{P}}
\newcommand{\pperp}{\perp \! \! \! \perp}
\newcommand{\Var}{\text{Var}}
\newcommand{\lam}{\lambda}
\newcommand{\convps}{\underset{n\rightarrow +\infty}{\overset{\text{p.s.}}{\longrightarrow}}}
\newcommand{\convpsm}{\underset{m\rightarrow +\infty}{\overset{\text{p.s.}}{\longrightarrow}}}
\newcommand{\convproba}{\underset{n\rightarrow +\infty}{\overset{\text{proba.}}{\longrightarrow}}}
\newcommand{\convloi}{\underset{n\rightarrow +\infty}{\overset{\text{d}}{\longrightarrow}}}
\newcommand{\convlp}{\underset{n\rightarrow +\infty}{\overset{L^p}{\longrightarrow}}}
\newcommand{\convldeux}{\underset{n\rightarrow +\infty}{\overset{L^2}{\longrightarrow}}}
\newcommand{\Z}{\mathbb{Z}}
\newcommand{\simU}{\sim \mathcal{U}([0;1])}
\newcommand{\Yh}{\hat{Y}}
\newcommand{\uh}{\hat{u}}
\newcommand{\Yb}{\bar{Y}}
\newcommand{\yb}{\bar{y}}
\newcommand{\Xb}{\bar{X}}
\newcommand{\uhb}{\bar{\hat{u}}}
\renewcommand{\b}{\beta}

\usepackage{enumitem}
% \setenumerate[1]{label={(\alph*)}} 



\usepackage{amsmath,amsfonts,amssymb,amsthm,mathtools} % AMS
\usepackage{icomma} 
\mathtoolsset{showonlyrefs=true} % Показывать номера только у тех формул, на которые есть \eqref{} в тексте.
%\usepackage{euscript}	 % Шрифт Евклид
%\usepackage{mathrsfs} % Красивый матшрифт
\usepackage{enumitem}
\usepackage{siunitx}
\usepackage{tikz} % To generate the plot from csv
\usepackage{pgfplots}

\usepackage{hyperref}

%%% Заголовок

% \newcommand{\latinword}[1]{\textsf{\itshape #1}}%


\usepackage{indentfirst}
\usepackage{fancyvrb}


\usepackage{fancyhdr} % весёлые колонтитулы
\pagestyle{fancy}
\lhead{Statistics 4MR}
\chead{}
\rhead{HA}
\lfoot{2021-2022}
\cfoot{}
\rfoot{\thepage/\pageref{LastPage}}
\renewcommand{\headrulewidth}{0.4pt}
\renewcommand{\footrulewidth}{0.4pt}



\usepackage{mathtools}


%\usepackage[noae]{Sweave}

\tolerance=10000
\emergencystretch=\maxdimen
\widowpenalty=10000
\clubpenalty=10000 
%\hyphenpenalty=10000
%\hbadness=10000

\usepackage{graphicx}
\usepackage{float} 

\usepackage{comment}
%Uncomment below to hide/show solution
%\includecomment{teacher}
\excludecomment{teacher}


\begin{document}
% \maketitle
	

\begin{enumerate}


\item  A/B testing. (25 points)

Consider two samples with sizes $n_x = n_y=50$ from:
\begin{itemize}
	\item two normal distributions with unit variances, means $\mu_x = 0$ and varying $\mu_y$;
	\item two normal distributions with variances one and two, means $\mu_x = 0$ and varying $\mu_y$;
	\item any other distributions (not normal) of your choice. 
	Here you should vary one parameter — location of the second distribution.
\end{itemize}

We would like to test the hypothesis $H_0$: $\mu_x = \mu_y$ against $H_a$: $\mu_x \neq \mu_y$ at nominal significance level $5\%$.
\begin{enumerate}
	\item For $\mu_y =0$ calculate the real significance level of $t$-test, Welch test and Mann-Whitney test. 
	Choose a grid of positive $\mu_y$ and calculate real power of these tests. Use 10000 simulations for each $\mu_y$.  
	\item For each case provide a plot that describes how 
	real significance level and real power do change with $\mu_y$.
	\item Which test is more appropriate in each case?
	\item How will the answers change with $n_x = n_y =1000$?
	
\end{enumerate}

\begin{teacher}
	
\textit{Solution:} 

	
\end{teacher}





\item Bootstrap (25 points)


Use the dataset \texttt{SIC33.csv} with the following variables:
\begin{itemize}
	\item \textit{output} — Value added.
	\item  \textit{labor} — Labor input.
	\item  \textit{capital} — Capital stock.
\end{itemize}

Obtain OLS estimates of the model 
\[
\ln Y_i = \beta_1 + \beta_K \ln K_i + \beta_L \ln L_i  + u_i.
\] 
Use the paired bootstrap with $B = 10000$.

\begin{enumerate}
	\item Obtain a bootstrap estimate of the standard error of $\hat \beta_K$. 
	
	\item Use this standard error estimate to test $H_0$: $\beta_K = 0$ against $H_a$: $\beta_K \neq 0$.
	
	\item Provide three 95\% CI for $\beta_K$: classic, heteroskedasticity robust (HC1), naive bootstrap.
% What happens if $u$ is heteroskedastic? 
	
	\item Provide 95\% bootstrap CI for the product $\beta_K \cdot \beta_L$.
	
	\item Estimate the same model using median (absolute value) regression. 
	The target function is 
	\[
	\min_{\hat\beta} \sum_{i=1}^n \abs{\ln Y_i  - \ln \hat Y_i}.
	\]
	Provide 95\% bootstrap CI for each $\beta$. 

\end{enumerate}







\begin{teacher}
	
\textit{Solution:} 

	
\end{teacher}



\newpage
\item CUPED. (25 points)

The data generating process is:
\[
\begin{cases}
n = 200; \\
x_i \sim \mathcal{N}(1; 1); \\
(u_i\mid x_i) \sim \mathcal{N}(0; 1); \\
(d_i \mid x_i, u_i) \sim \mathrm{Bernoulli}(1/2); \\
y_i = 3 + 2 x_i^2 + \delta d_i + u_i \\
\end{cases}
\]

Here $x_i$ is a characterestic of an individual, $d_i = 1$ for those in experimental group, 
and $d_i=0$ for those in control group, $y_i$ is the target variable. 
We are interested in estimating the effect $\delta$. 
Consider three strategies:

\begin{itemize}
	\item SIMPLE. Regression with a standard confidence interval for $\delta$:
	\[
		\hat y_i = \hat \beta_1 + \hat \delta d_i;
	\]
	\item CUPED-A. Two stages:
	\begin{enumerate}
		\item[Step I.] Estimate regression and calculate residual $r_i = y_i - \hat y_i$.
		\[
			\hat y_i = \hat \beta_1 + \hat \beta_2 x_i;
		\]
		\item[Step II.] Regress $r_i$ on $d_i$ and build a standard CI for $\delta$:
		\[
		\hat r_i = \hat \alpha + \hat \delta d_i.	
		\]
	\end{enumerate}
	\item CUPED-B. Two stages:
	\begin{enumerate}
		\item[Step I.] Estimate regression and calculate quasi-residual $r_i = y_i - (\hat\beta_1 + \hat\beta_2 x_i)$.
		\[
			\hat y_i = \hat \beta_1 + \hat \beta_2 x_i + \hat \delta d_i;
		\]
		\item[Step II.] Regress $r_i$ on $d_i$ and build a standard CI for $\delta$:
		\[
		\hat r_i = \hat \alpha + \hat \delta d_i.	
		\]
	\end{enumerate}
\end{itemize}

Your task is:
\begin{enumerate}
	\item Simulate the data set 1000 times for each of these deltas: $\delta = 0$, $\delta = 0.1$, $\delta = 1$, $\delta = 10$.
	\item Calculate 95\% nominal CI according to SIMPLE, CUPED-A and CUPED-B.
	\item For $\delta =0$ provide the histogram of p-values to check $H_0$: $\delta = 0$ agains $\delta \neq 0$. 
	\item What is the actual coverage probability for each method for each delta?
	\item For each delta for each method find: the average length of the interval, 
	the number of times the interval was the shortest among three methods considered. 
\end{enumerate}


\end{enumerate}








\end{document}

