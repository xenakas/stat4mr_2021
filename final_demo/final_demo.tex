% arara: xelatex
\documentclass[12pt]{article}

\usepackage{tikz} % картинки в tikz
\usepackage{microtype} % свешивание пунктуации

\usepackage{array} % для столбцов фиксированной ширины

\usepackage{indentfirst} % отступ в первом параграфе

\usepackage{sectsty} % для центрирования названий частей
\allsectionsfont{\centering}

\usepackage{amsmath, amssymb, amsthm} % куча стандартных математических плюшек

\usepackage{amsfonts}

\usepackage{comment}

\usepackage[top=2cm, left=1.2cm, right=1.2cm, bottom=2cm]{geometry} % размер текста на странице

\usepackage{lastpage} % чтобы узнать номер последней страницы

\usepackage{enumitem} % дополнительные плюшки для списков
%  например \begin{enumerate}[resume] позволяет продолжить нумерацию в новом списке
\usepackage{caption}


\usepackage{hyperref} % гиперссылки

\usepackage{multicol} % текст в несколько столбцов


\usepackage{fancyhdr} % весёлые колонтитулы
\pagestyle{fancy}
\lhead{Final demo}
\chead{DON'T PANIC}
\rhead{2021-12-27}
\lfoot{}
\cfoot{}
\rfoot{}
\renewcommand{\headrulewidth}{0.4pt}
\renewcommand{\footrulewidth}{0.4pt}

\let\P\relax
\DeclareMathOperator{\P}{\mathbb{P}}
\DeclareMathOperator{\Cov}{Cov}
\DeclareMathOperator{\E}{\mathbb{E}}
\DeclareMathOperator{\Var}{Var}
\DeclareMathOperator{\Corr}{Corr}
\DeclareMathOperator{\plim}{plim}
\DeclareMathOperator{\pCorr}{pCorr}


\newcommand{\cN}{\mathcal{N}}

\usepackage{todonotes} % для вставки в документ заметок о том, что осталось сделать
% \todo{Здесь надо коэффициенты исправить}
% \missingfigure{Здесь будет Последний день Помпеи}
% \listoftodos - печатает все поставленные \todo'шки


% более красивые таблицы
\usepackage{booktabs}
% заповеди из докупентации:
% 1. Не используйте вертикальные линни
% 2. Не используйте двойные линии
% 3. Единицы измерения - в шапку таблицы
% 4. Не сокращайте .1 вместо 0.1
% 5. Повторяющееся значение повторяйте, а не говорите "то же"



\usepackage{fontspec}
\usepackage{polyglossia}

\setmainlanguage{english}
\setotherlanguages{russian}

% download "Linux Libertine" fonts:
% http://www.linuxlibertine.org/index.php?id=91&L=1
\setmainfont{Linux Libertine O} % or Helvetica, Arial, Cambria
% why do we need \newfontfamily:
% http://tex.stackexchange.com/questions/91507/
\newfontfamily{\cyrillicfonttt}{Linux Libertine O}

\AddEnumerateCounter{\asbuk}{\russian@alph}{щ} % для списков с русскими буквами
% \setlist[enumerate, 2]{label=\asbuk*),ref=\asbuk*}


\begin{document}

Rules: online test in lms, no proctoring, 20 questions, 60 minutes, only numerical answers are checked, 
two digits after decimal point are requested, 
use anything you want (calculators, python/r code, google, \ldots), don't cheat. 


\begin{enumerate}
    \item (bootstrap) I have a sample $X_1$, \ldots, $X_{100}$.
    
    I generate one naive bootstrap sample $X^{*}_{1}$, \ldots, $X^{*}_{100}$. 

    What is the probability that the first observation will be present in the bootstrap sample 2 times or more?

    \item (bootstrap) Nature generates random variables $X_1$, \ldots, $X_{100}$ independently and uniformly on $[0;10]$.
    
    I generate one naive bootstrap sample $X_1^*$, \ldots, $X_{100}^*$. 

    Find the variance $\Var(X_1^*)$.
    \item (welch) We have data for an $AB$-experiment $\bar X_a = 10$, $\bar X_b = 12$, 
    $n_a = 20$, $n_b = 30$, $\sum (X_i^a - \bar X_a)^2 = 100$, $\sum (X_i^b - \bar X_b)^2 = 200$.

    Calculate the standard error of $\bar X_a - \bar X_b$ for the Welch test. 

    \item (welch) Assume that $X_i$ are independent and identically normally distributed $\cN(\mu, \sigma^2)$, 
    sample size is $n=10$. 
    
    Find $\Var(\sum (X_i - \bar X)^2 / (n - 1))$.
    
    \item (mw test) I have five results of two runners $A$ and $B$ for the 5 km race: 25:12 (A), 26:34 (B), 27:43 (A), 28:12 (A), 29:05 (B).
    
    Calculate Mann-Whitney statistic $U_A$ that tests the null-hypothesis of equal distributions of time. 
    
    (The statistic $U_A$ should positively depend on the ranks of the runner $A$). 
    \item (mw test) I have five results of two runners $A$ and $B$ for the 5 km race: three results for $A$ and two results for $B$.
    Assume that the running time for both runners are continuously distributed and their distribution are equal. 

    What is the probability that the running times of the runner $A$ will get the ranks 1 and 5?
    
    \item (cuped) Consider three variables: target variable $y_i$, predictior $x_i$ and indicator of treatment $z_i \in \{0,1\}$.
    The treatment $z_i$ was assigned independently of $x_i$, total $n=200$. 

    The matrix of all cross products (sums of the form $\sum a_i b_i$) $C$ is provided. 
    The order of variables is $y$, $1$, $x$ and $z$. For example, $\sum 1 \times y_i = 10$:

    \[
    C = \begin{pmatrix}
        500 & 10 & 2 & 8 \\
        10 & 200 & 100 & 100 \\
        2 &  100 & 100 &  40   \\
        8 &  100 & 40  & 100 \\
    \end{pmatrix}    
    \]
    
    Consider CUPED with first regression given by $\hat y_i = \hat\alpha_1 + \hat\alpha_2 x_i$ with residuals $r_i = y_i - \hat y_i$.

    What is the cross-product $\sum r_i z_i$?


    \item (cuped) Consider three variables: target variable $y_i$, predictior $x_i$ and indicator of treatment $z_i \in \{0,1\}$.
    The treatment $z_i$ was designed to be independent of $x_i$, but in fact $x_i = f_i \cdot (1 + 0.01 z_i)$.

    We suppose that $z_i$ are Bernoulli with $p=0.5$, $f_i \sim \cN(1;1)$ and they are independent.

    Find the probability limit 
    \[
    \plim \frac{\sum (x_i - \bar x)(z_i - \bar x)}{n-1}.    
    \]

    \item (matching) Vasiliy uses knn with 1 neighbour to match observations. 
    Here $z$ is treatment indicator, $x$ is predictor and $y$ is target variable. 

    \begin{tabular}{cccc}
    \toprule
    $i$ & $y_i$ & $x_i$ & $z_i$ \\
    \midrule
    1 & 6 & 6 & 0 \\
    2 & 6 & 1 & 0 \\
    3 & 6 & 3 & 0 \\
    4 & 6 & 7 & 0 \\
    5 & 6 & 2 & 1 \\
    6 & 6 & 5 & 1 \\
    7 & 6 & 9 & 1 \\        
    8 & 6 & 1 & 1 \\
    \bottomrule
    \end{tabular}

    Which indivial will be matched with individual number 3?
    

    \item (matching) The indicator of treatment $z$ is Bernoulli with $p=0.5$. 
    The conditional distribution of variables $y(0)$ (hypothetical outcome if $z =0$) and $y(1)$ (hypothetical outcome if $z=1$) 
    is given in two tables

    \begin{tabular}{ccc}
    \toprule
    Condition $z = 0$ & $y(0) = 0$ & $y(0) = 1$ \\
    \midrule
    $y(1)=0$ & 0.1 & 0.5 \\
    $y(1)=1$ & 0.3 & 0.1 \\
    \bottomrule
    \end{tabular}

    \begin{tabular}{ccc}
        \toprule
        Condition $z = 1$ & $y(0) = 0$ & $y(0) = 1$ \\
        \midrule
        $y(1)=0$ & 0.4 & 0.1 \\
        $y(1)=1$ & 0.2 & 0.3 \\
        \bottomrule
    \end{tabular}

    What is the average treatment effect $\E(y(1) = y(0))$?


    \item (multiple comparison) I do 100 independent tests at significance level 5\%. 
    The null hypothesis for all 100 tests is actually true, but I don't know this. 
    
    What is the probability that I will receive at least 2 significant results?
    
    \item (multiple comparison) I have 100 hypothesis with independent statistics. 
    The null hypothesis for all 100 cases is actually true, but I don't know this. 
    
    I calculate all p-values. 
    If the two lowes p-value are both lower than $0.05$ I wrongly conclude that not all $H_0$ are true. 
    Otherwise I correctly conclude that all $H_0$ are true. 

    What is the probability that I will get the correct conclusion?
    
    \item (sample size) My target variable $y$ is continuous. 
    I have two stratas of respondents with preliminary estimates $\hat\sigma_s$, 
    total strata size $N_s$, and cost per one observation $c_s$.

    My total budget is 5000. I wish to estimate $\E(y_i)$. 
    
    How much observations I should sample from the first strata?

    \begin{tabular}{cccc}
        \toprule
        Strata $s$ & $\hat\sigma_s$ & $N_s$ & $c_s$ \\
        \midrule
        1 & 20 & $10^5$ & 4 \\
        2 & 30 & $2\cdot 10^5$ & 1\\
        \bottomrule
    \end{tabular}
    
    
    \item (sample size) My target variable is binary and I wish 
    minimal detectable effect equal to $0.01$, probability of I-error  not greater than $0.02$, 
    probability of II-error not greater than $0.10$, control and experimental group of the same size equal to $n$.

    What is minimal value of $n$?

    \item (contingency table) I eated 10 M\&Ms: 2 green, 1 red, 4 yellow, 1 green, 2 red.  
    
    Only these three colors are possible. I assume that yellow and green colors are equally probable.  
    
    Calculate the maximal log likelihood for my model.  
    \item (contingency table) Consider the following contingency table 
    
    \begin{tabular}{@{}lll@{}}
        \toprule
         & $B=1$ & $B=2$ \\ 
         \midrule
         $A=1$ & 10 & 20 \\
         $A=2$ & 30 & 40 \\
        \bottomrule
        \end{tabular}


    Calculate $LR$ statistic that checks the hypothesis that $A$ and $B$ are independent against dependency alternative. 

    \item (anova 1+2) Vasiliy loves to eat shaurma. He has three local shaurma dealers. Vasiliy bought 7 shaurmas from each dealer. 
    and measured their weight. He would like to test the hypothesis that mean weight is the same for all dealers. 

    Total sum of squares is 1000, between sum of squares is 500. 

    Calculate the $F$-statistic to test the hypothesis.
    
    \item (anova 1+2) Vasiliy loves to eat shaurma. He has three local shaurma dealers with two types of shaurma.  
    Vasiliy bought 7 shaurmas of each type from each dealer 
    and measured their weight. He would like to test the hypothesis that weight depenends on the shaurma type and not on the dealer.
    He assumes no interaction.  

    Total sum of squares is 1000, sum of squares explained by dealers is 300, sum of squares explained by type is 500. 

    Calculate the $F$-statistic to test the hypothesis.

    \item (partial correlation) The variables $X$ and $Y$ are jointly normal with zero means,
unit variances and $\Corr(X, Y) = 0.8$. 

Find $\alpha$ such that $X^* = X - \alpha Y$ is not correlated with $Y$. 

\item (partial correlation) The variables $X_1$, $X_2$, \ldots{ } are independent and identically distributed with mean 5 and variance 7. 

Find $\pCorr(X_1, X_2; S)$ where $S = X_2 + X_3$.
    
    
\end{enumerate}


\end{document}
