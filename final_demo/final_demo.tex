% arara: xelatex
\documentclass[12pt]{article}

\usepackage{tikz} % картинки в tikz
\usepackage{microtype} % свешивание пунктуации

\usepackage{array} % для столбцов фиксированной ширины

\usepackage{indentfirst} % отступ в первом параграфе

\usepackage{sectsty} % для центрирования названий частей
\allsectionsfont{\centering}

\usepackage{amsmath, amssymb, amsthm} % куча стандартных математических плюшек

\usepackage{amsfonts}

\usepackage{comment}

\usepackage[top=2cm, left=1.2cm, right=1.2cm, bottom=2cm]{geometry} % размер текста на странице

\usepackage{lastpage} % чтобы узнать номер последней страницы

\usepackage{enumitem} % дополнительные плюшки для списков
%  например \begin{enumerate}[resume] позволяет продолжить нумерацию в новом списке
\usepackage{caption}


\usepackage{hyperref} % гиперссылки

\usepackage{multicol} % текст в несколько столбцов


\usepackage{fancyhdr} % весёлые колонтитулы
\pagestyle{fancy}
\lhead{Stochastic Calculus 2021}
\chead{DON'T PANIC}
\rhead{2021-12-21}
\lfoot{Short rules: 120 minutes, A4 cheat sheet and calculator are allowed, $W_t$ always denotes a Wiener process.}
\cfoot{}
\rfoot{}
\renewcommand{\headrulewidth}{0.4pt}
\renewcommand{\footrulewidth}{0.4pt}

\let\P\relax
\DeclareMathOperator{\P}{\mathbb{P}}
\DeclareMathOperator{\Cov}{Cov}
\DeclareMathOperator{\E}{\mathbb{E}}
\DeclareMathOperator{\Var}{Var}
\DeclareMathOperator{\Corr}{Corr}
\DeclareMathOperator{\pCorr}{pCorr}


\newcommand{\cN}{\mathcal{N}}

\usepackage{todonotes} % для вставки в документ заметок о том, что осталось сделать
% \todo{Здесь надо коэффициенты исправить}
% \missingfigure{Здесь будет Последний день Помпеи}
% \listoftodos - печатает все поставленные \todo'шки


% более красивые таблицы
\usepackage{booktabs}
% заповеди из докупентации:
% 1. Не используйте вертикальные линни
% 2. Не используйте двойные линии
% 3. Единицы измерения - в шапку таблицы
% 4. Не сокращайте .1 вместо 0.1
% 5. Повторяющееся значение повторяйте, а не говорите "то же"



\usepackage{fontspec}
\usepackage{polyglossia}

\setmainlanguage{english}
\setotherlanguages{russian}

% download "Linux Libertine" fonts:
% http://www.linuxlibertine.org/index.php?id=91&L=1
\setmainfont{Linux Libertine O} % or Helvetica, Arial, Cambria
% why do we need \newfontfamily:
% http://tex.stackexchange.com/questions/91507/
\newfontfamily{\cyrillicfonttt}{Linux Libertine O}

\AddEnumerateCounter{\asbuk}{\russian@alph}{щ} % для списков с русскими буквами
% \setlist[enumerate, 2]{label=\asbuk*),ref=\asbuk*}


\begin{document}

Rules: online test in lms, no proctoring, 20 questions, 60 minutes, only numerical answers are checked, 
two digits after decimal point are requested, 
use anything you want (calculators, python code, google, \ldots), don't cheat. 


\begin{enumerate}
    \item (bootstrap) I have a sample $X_1$, \ldots, $X_{100}$.
    
    I generate one naive bootstrap sample $X_1^*$, \ldots, $X_{100}^*$. 

    What is the probability that the first observation will be present in the bootstrap sample 2 times or more?

    \item (bootstrap) Nature generates random variables $X_1$, \ldots, $X_{100}$ independently and uniformly on $[0;10]$.
    
    I generate one naive bootstrap sample $X_1^*$, \ldots, $X_{100}^*$. 

    Find the variance $\Var(X_1^*)$.
    \item (welch)  
    \item (welch)
    \item (mw test)
    \item (mw test)
    \item (matching)
    \item (matching)
    \item (multiple comparison)
    \item (multiple comparison)
    \item (sample size)
    \item (sample size)
    \item (contingency table) I eated 10 M\&Ms: 2 green, 1 red, 4 yellow, 1 green, 2 red.  
    
    Only these three colors are possible. I assume that yellow and green colors are equally probable.  
    
    Calculate the maximal log likelihood for my model.  
    \item (contingency table) Consider the following contingency table 
    
    \begin{tabular}{@{}lll@{}}
        \toprule
         & $B=1$ & $B=2$ \\ 
         \midrule
         $A=1$ & 10 & 20 \\
         $A=2$ & 30 & 40 \\
        \bottomrule
        \end{tabular}


    Calculate $LR$ statistic that checks the hypothesis that $A$ and $B$ are independent against dependency alternative. 

    \item (anova 1+2)
    \item (anova 1+2)

    \item (partial correlation) The variables $X$ and $Y$ are jointly normal with zero means,
unit variances and $\Corr(X, Y) = 0.8$. 

Find $\alpha$ such that $X^* = X - \alpha Y$ is not correlated with $Y$. 

\item (partial correlation) The variables $X_1$, $X_2$, \ldots{ } are independent and identically distributed with mean 5 and variance 7. 

Find $\pCorr(X_1, X_2; S)$ where $S = X_2 + X_3$.
    
    
\end{enumerate}


\end{document}
